%PREAMBLE
\documentclass{article}% use option titlepage to get the title on a page of its own.
\usepackage{blindtext} % a package for creating random text
\usepackage{listings} % a package to embed code
\usepackage{color} 

\definecolor{dkgreen}{rgb}{0,0.6,0}
\definecolor{gray}{rgb}{0.5,0.5,0.5}
\definecolor{mauve}{rgb}{0.58,0,0.82}

\lstset{frame=tb,
  language=Python,
  aboveskip=3mm,
  belowskip=3mm,
  showstringspaces=false,
  columns=flexible,
  basicstyle={\small\ttfamily},
  numbers=none,
  numberstyle=\tiny\color{gray},
  keywordstyle=\color{blue},
  commentstyle=\color{dkgreen},
  stringstyle=\color{mauve},
  breaklines=true,
  breakatwhitespace=true,
  tabsize=3
}

\title{Logging}
\date{Sep 20th, 2021}
\author{Matt Waismann}

%MAIN
\begin{document}
\maketitle
Without logging, most people check the program behaved correctly by embedding print statements at various points in the script. The better practice is to use \textit{logging}. \\

There are five standard logging levels:
\begin{itemize}
  \item \# DEBUG: Detailed information, typically of interest only when diagnosing problems.
  \item \# INFO: Confirmation that things are working as expected.
  \item \# WARNING: An indication that something unexpected happened, or indicative of some problem in the near future (e.g. 'disk space low'). The software is still working as expected.
  \item \# ERROR: Due to a more serious problem, the software has not been able to perform some function.
  \item \# CRITICAL: A serious error, indicating that the program itself may be unable to continue running.
\end{itemize}

\begin{lstlisting}
    print("Hello World")
\end{lstlisting}
\end{document}